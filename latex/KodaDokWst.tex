%*******************************************************************************
% Definicje stylu dokumentu
%*******************************************************************************

%===============================================================================
% klasa dokumentu

%\documentclass[12pt, a4paper, twoside, titlepage, final]{mwbk}
%\documentclass[10pt,a4paper,onecolumn,oneside,11pt,wide,floatssmall]{book}
\documentclass[10pt,a4paper,onecolumn,oneside,11pt,wide,floatssmall]{article}


%===============================================================================
% Pakiety
%\usepackage[latin2]{inputenc}
\usepackage{polski}
%\usepackage[cp1250]{inputenc}
\usepackage[utf8]{inputenc}				% kodowanie �r�d�a
\usepackage[polish]{babel}				% polskie przenoszenie wyraz�w (hyph.)
\usepackage[T1]{fontenc}					% font PL
\usepackage{url}								% polecenie \url
\usepackage{amsfonts}						% fonty matematyczne
\usepackage{graphicx}						% wstawianie grafiki
\usepackage{color}							% kolory
\usepackage{fancyhdr}						% paginy g�rne i dolne
\usepackage[plainpages=false]{hyperref}		% dynamiczne linki
\usepackage{calc}							% operacje arytmetyczne w TeX'u
\usepackage{tabularx}						% rozci�gliwe tabele
\usepackage{array}							% standardowe tabele
\usepackage{geometry}
\usepackage{hyperref}
\usepackage{subfigure}
\usepackage{wrapfig}
\usepackage{indentfirst}
\usepackage{amsmath}
\usepackage{color}
\usepackage{array}
\usepackage{pdflscape}
\usepackage{amsmath}
\usepackage{textcomp}
\usepackage[font={small,it}]{caption}
\usepackage{etoolbox}


\linespread{1.3}								% 1.3 do interlinii 1.5


\patchcmd{\thebibliography}{\chapter*}{\section*}{}{}

\bibliographystyle{plain}
% w�asne pakiety

%===============================================================================
% Ustawienia dokumentu

\frenchspacing

% ustawienia wymiar�w
\oddsidemargin 0mm							% margines nieparzystych stron
\evensidemargin 0mm							% margines parzystych stron
\headheight 15pt								% wysoko�� paginy g�rnej
\topmargin 0mm									% margines g�rny
\setlength{\parindent}{0pt}
\setlength{\parskip}{1ex plus 0.5ex minus 0.2ex}
% styl paginacji
\pagestyle{fancy}
% \renewcommand{\chaptermark}[1]{}%{\markboth{#1}{}} % BO ARTICLE
\renewcommand{\sectionmark}[1]{}%{\markright{\thesection\ #1}{}}
\renewcommand{\thesection}{\arabic{section}}


% nag��wek 
\fancyhf{}
\fancyhead[R]{\thepage}
\fancyhead[L]{[KODA]~Kodowanie różnicowe + koder Huffmana (dokumentacja wstępna)}
%\fancyhead[LO]{\small\nouppercase{\rightmark}}
%\fancyhead[R]{\small\nouppercase{\leftmark}}
\renewcommand{\headrulewidth}{0.1pt}
\renewcommand{\footrulewidth}{0pt}

% nag��wek w stylu plain 
\fancypagestyle{plain}
{
\fancyhf{}
\renewcommand{\headrulewidth}{0pt}
\renewcommand{\footrulewidth}{0pt}
}

% ta sekwencja tworzy czyste kartki na stronach po \cleardoublepage
\makeatletter
\def\cleardoublepage{\clearpage\if@twoside \ifodd\c@page\else
	\hbox{}
	\vspace*{\fill}
	\thispagestyle{empty}
	\newpage
	\if@twocolumn\hbox{}\newpage\fi\fi\fi}
\makeatother

%===============================================================================
% Zmienne �rodowiskowe i polecenia

% definicja
% \newtheorem{definition}{Definicja}[chapter] % BO CHAPTER
\newtheorem{definition}{Definicja}

% twierdzenie
% \newtheorem{theorem}{Twierdzenie}[chapter] % BO CHAPTER
\newtheorem{theorem}{Twierdzenie}

% obcoj�zyczne nazwy
\newcommand{\foreign}[1]{\emph{#1}}

% pozioma linia
\newcommand{\horline}{\noindent\rule{\textwidth}{0.4mm}}

% wstawianie obrazk�w {plik}{caption}{opis}
\newcommand{\fig}[3]
{
\begin{figure}[!htb]
\begin{center}
\includegraphics[width=\textwidth]{#1}
\caption[#2]{#2. #3}
\label{#1}
\end{center}
\end{figure}
}

%===============================================================================
% ustawienia pakietu hyperref

\hypersetup
{
%colorlinks=true,			% false: boxed links; true: colored links
%linkcolor=black,			% color of internal links
%citecolor=black,			% color of links to bibliography
%filecolor=black,			% color of file links
%urlcolor=black			% color of external links
}

%===============================================================================



\title{Kompresja Danych \\ \Huge{Kodowanie różnicowe + koder Huffmana} \\ \Large{Dokumentacja wstępna} }
\author{ Piotr Chmielewski \\ Michał Dobrzański \\ Maciej Janusz Krajsman \\ Marcin Lembke \\ \\ Politechnika Warszawska, \\ Wydział Elektroniki i Technik Informacyjnych.}

\begin{document}
\maketitle    

\section{Założenia projektowe}
\label{sec:zalozenia_projektowe}

\subsection{Zadanie projektowe}
\label{subsec:zadanie_projektowe}
Opracować algorytm kodowania predykcyjnego (pozycje \cite{Przelaskowski}, \cite{Sayood} literatury uzupełniającej do wykładu) danych dwuwymiarowych wykorzystując do predykcji: lewego sąsiada, górnego sąsiada, medianę lewego, lewego-górnego, górnego sąsiada. Wyznaczyć histogramy danych różnicowych dla danych wejściowych o rozkładzie równomiernym, normalnym, Laplace'a oraz wybranych obrazów testowych. Zakodować dane różnicowe przy użyciu klasycznego algorytmu Huffmana. Wyznaczyć entropię danych wejściowych i różnicowych, porównać ze średnią długością bitową kodu wyjściowego. Ocenić efektywność algorytmu do kodowania obrazów naturalnych. 

\subsection{Narzędzia programistyczne}
\label{subsec:narzedzia_programistyczne}
Projekt napisany zostanie w~języku Python, z~użyciem potrzebnych bibliotek (np. $Pillow$~---~konkretne decyzje w~tej kwestii zapadną na etapie implementacji). Wykorzystane zostanie środowisko $JetBrains$ $PyCharm$ Community Edition.

\section{Metody kodowania}
\label{sec:metody}

\subsection{Kodowanie predykcyjne}
\label{subsec:kodowanie_roznicowe}

Kodowanie predykcyjne pozwala zredukować rozmiar danych dzięki wykorzystaniu wiedzy o~już przetworzonej części informacji. 

\subsection{Kodowanie  Huffmana}
\label{subsec:kod_huffamana}

Kodowanie Huffmana jest bezstratną metodą, pozwalającą otrzymać efektywny kod symboli. Uzyskany kod jest optymalnym kodem prefiksowym, tj. nie istnieje żaden inny kod w~tej kategorii, który zapewniałby mniejszą średnią długość słowa kodowego. Idea tej metody opiera się na dwóch założeniach:

\begin{enumerate} \itemsep1pt
	\item Długość słowa kodowego dla danego symbolu jest tym mniejsza, im częściej występuje on w~alfabecie;
	\item Dwa symbole o~najmniejszej częstości występowania w~alfabecie mają słowa kodowe o~równej długości.
\end{enumerate}

\section{Testowanie}
\label{sec:testowanie}

\subsection{Metody oceny efektywności kompresji danych}
\label{subsec:metody_oceny_efektywnosci_kompresji_danych}
Ocena efektywności zaimplementowanych metod kodowania polegać będzie na porównaniu:
\begin{itemize} \itemsep1pt
	\item entropii
	\item średniej długości słowa kodowego
\end{itemize}
danych wejściowych i~wyjściowych dla różnych obrazów testowych. Zaprezentowane zostaną również histogramy ich oraz danych różnicowych dla każdej metody kodowania.

\subsection{Dane testowe}
\label{subsec:dane_testowe}

Zbiór danych testowych będzie składał się z~kilku obrazów naturalnych, a~także wygenerowanych przez program trzech losowych obrazów, których wartości natężeń pikseli reprezentować będą:

\begin{itemize} \itemsep1pt
	\item rozkład równomierny (równoważny szumowi białemu)
	\item rozkład normalny (Gaussa)
	\item rozkład Laplace'a
\end{itemize}

%*******************************************************************************
% Bibliografia - spis literatury wykorzystanej przy tworzeniu pracy
%*******************************************************************************

\begin{thebibliography}{99}
\addcontentsline{toc}{chapter}{Bibliografia}

%1
\bibitem{Przelaskowski} Przelaskowski Artur, \emph{,,Kompresja danych: podstawy, metody bezstratne, kodery obraz\'{o}w''}, 
Wyd. I, Warszawa, Wyd. BTC, 2005, ISBN: 83-60233-05-5.

%2
\bibitem{Sayood} Sayood Khalid, \emph{,,Kompresja danych, wprowadzenie''}, Wyd. I, Warszawa, Wyd. RM, 2002, ISBN: 83-7243-094-2.


\end{thebibliography}

\clearpage

%===============================================================================

\end{document}

